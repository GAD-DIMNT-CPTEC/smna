%% Generated by Sphinx.
\def\sphinxdocclass{report}
\documentclass[letterpaper,10pt,english]{sphinxmanual}
\ifdefined\pdfpxdimen
   \let\sphinxpxdimen\pdfpxdimen\else\newdimen\sphinxpxdimen
\fi \sphinxpxdimen=.75bp\relax

\usepackage[utf8]{inputenc}
\ifdefined\DeclareUnicodeCharacter
 \ifdefined\DeclareUnicodeCharacterAsOptional
  \DeclareUnicodeCharacter{"00A0}{\nobreakspace}
  \DeclareUnicodeCharacter{"2500}{\sphinxunichar{2500}}
  \DeclareUnicodeCharacter{"2502}{\sphinxunichar{2502}}
  \DeclareUnicodeCharacter{"2514}{\sphinxunichar{2514}}
  \DeclareUnicodeCharacter{"251C}{\sphinxunichar{251C}}
  \DeclareUnicodeCharacter{"2572}{\textbackslash}
 \else
  \DeclareUnicodeCharacter{00A0}{\nobreakspace}
  \DeclareUnicodeCharacter{2500}{\sphinxunichar{2500}}
  \DeclareUnicodeCharacter{2502}{\sphinxunichar{2502}}
  \DeclareUnicodeCharacter{2514}{\sphinxunichar{2514}}
  \DeclareUnicodeCharacter{251C}{\sphinxunichar{251C}}
  \DeclareUnicodeCharacter{2572}{\textbackslash}
 \fi
\fi
\usepackage{cmap}
\usepackage[T1]{fontenc}
\usepackage{amsmath,amssymb,amstext}
\usepackage{babel}
\usepackage{times}
\usepackage[Sonny]{fncychap}
\usepackage[dontkeepoldnames]{sphinx}

\usepackage{geometry}

% Include hyperref last.
\usepackage{hyperref}
% Fix anchor placement for figures with captions.
\usepackage{hypcap}% it must be loaded after hyperref.
% Set up styles of URL: it should be placed after hyperref.
\urlstyle{same}

\addto\captionsenglish{\renewcommand{\figurename}{Fig.}}
\addto\captionsenglish{\renewcommand{\tablename}{Table}}
\addto\captionsenglish{\renewcommand{\literalblockname}{Listing}}

\addto\captionsenglish{\renewcommand{\literalblockcontinuedname}{continued from previous page}}
\addto\captionsenglish{\renewcommand{\literalblockcontinuesname}{continues on next page}}

\addto\extrasenglish{\def\pageautorefname{page}}





\title{Inctime Documentation}
\date{Nov 14, 2019}
\release{V0.01}
\author{João Gerd Zell de Mattos}
\newcommand{\sphinxlogo}{\vbox{}}
\renewcommand{\releasename}{Release}
\makeindex

\begin{document}

\maketitle
\sphinxtableofcontents
\phantomsection\label{\detokenize{index::doc}}


Inctime is a Fortran program to calculate dates. It can convert between date formats and output in a variety of ways, being suitable to use within fortran based modeling applications.


\chapter{Main routine and modules}
\label{\detokenize{index:main-routine-and-modules}}\label{\detokenize{index:inctime-documentation}}

\section{Main}
\label{\detokenize{main:main}}\label{\detokenize{main::doc}}
The following code is related to the inctime core.


\subsection{Inctime}
\label{\detokenize{inctime:inctime}}\label{\detokenize{inctime::doc}}
Inctime is the main program.
\paragraph{Program}
\index{inctime (fortran program)|textbf}

\begin{fulllineitems}
\phantomsection\label{\detokenize{inctime:f/inctime}}\pysigline{\sphinxbfcode{program  }\sphinxbfcode{inctime}}
Main program
\begin{description}
\item[{Package Overview}] \leavevmode
Inctime is a set of routines in fortran 90 to perform
calculations with dates; The main function is to calculate
dates to past or to future from any date. Other operations can
be performed by using the modules available in this routine.
Among other duties, they are calculating the Julian day, number
of years, months, days and hours between any two dates.

\item[{Autor}] \leavevmode
João Gerd Zell de Mattos

\item[{Affiliation}] \leavevmode
Group on Data Assimilation Development - CPTEC/INPE

\item[{Date}] \leavevmode
March 22, 2011

\item[{Usage}] \leavevmode
The way to use this program is through the command line as follows:

inctime {[}yyyymmddhh, yyyymmdd{]} {[}\textless{}+,-\textgreater{}nynmndnhnnns{]} {[}Form. output{]}

\end{description}

where
\begin{quote}
\begin{itemize}
\item {} \begin{description}
\item[{{[}yyyymmddhh, yyyymmdd{]}}] \leavevmode\begin{itemize}
\item {} 
Initial Time

\end{itemize}

\end{description}

\item {} \begin{description}
\item[{{[}\textless{}+,-\textgreater{}nynmndnhnnns{]}}] \leavevmode\begin{itemize}
\item {} 
( -) calculate the passed date

\item {} 
( +) calculate the future date (default)

\item {} 
(ny) Number of year (default is 0)

\item {} 
(nm) Number of months (default is 0)

\item {} 
(nd) Number of days (default is 0)

\item {} 
(nh) Number of hours (default is 0)

\item {} 
(nn) Number of minutes (default is 0)

\item {} 
(ns) Number of seconds (default is 0)

\end{itemize}

\end{description}

\item {} \begin{description}
\item[{{[} Form. Output {]}}] \leavevmode\begin{itemize}
\item {} 
Format to output date. is a template Format
The format descriptors are similar to those
used in the GrADS.

\end{itemize}
\begin{itemize}
\item {} 
“\%y4”  substitute with a 4 digit year

\item {} 
“\%y2”  a 2 digit year

\item {} 
“\%m1”  a 1 or 2 digit month

\item {} 
“\%m2”  a 2 digit month

\item {} 
“\%mc”  a 3 letter month in lower cases

\item {} 
“\%Mc”  a 3 letter month with a leading letter in upper case

\item {} 
“\%MC”  a 3 letter month in upper cases

\item {} 
“\%d1”  a 1 or 2 digit day

\item {} 
“\%d2”  a 2 digit day

\item {} 
“\%h1”  a 1 or 2 digit hour

\item {} 
“\%h2”  a 2 digit hour

\item {} 
“\%h3”  a 3 digit hour (?)

\item {} 
“\%n2”  a 2 digit minute

\item {} 
“\%e”   a string ensemble identify

\item {} 
“\%jd”  a julian day without hours decimals

\item {} 
“\%jdh” a julian day with hour decimals

\item {} 
“\%jy”  a day of current year without hours decimals

\item {} 
“\%jyh” a day of current year with hours decimals

\end{itemize}

\end{description}

\end{itemize}

Can use words to compose the output format.
\end{quote}
\begin{description}
\item[{Examples}] \leavevmode\begin{itemize}
\item {} 
inctime 2001091000 +1d \%d2/\%m2/\%y4

\item {} 
inctime 2001091000 +48h30n \%h2Z\%d2\%MC\%y4

\item {} 
inctime 2001091000 -1h30n 3B42RT.\%y4\%m2\%d2\%h2.bin

\item {} 
inctime 2001091000 -2h45n 3B42RT.\%y4\%m2\%d2\%h2.bin

\item {} 
inctime 2001091000 -1y3m2d1h45n ANYTHING.\%y4\%m2\%d2\%h2.ANYTHING

\end{itemize}

\item[{History}] \leavevmode\begin{itemize}
\item {} 
22 Mar 2011 - Joao Gerd - Initial Code

\item {} 
27 Jul 2011 - Joao Gerd - Bug into the end print

\item {} 
19 Jul 2012 - Joao Gerd - Add option to month increment/decrement

\item {} 
15 Apr 2013 - Joao Gerd - correct bug to incr/decr month

\item {} \begin{description}
\item[{03 May 2013 - Joao Gerd - correct bug to incr/decr all times (ym was not}] \leavevmode
being properly initialized)

\end{description}

\item {} 
05 Feb 2014 - Joao Gerd - Add julian day

\item {} \begin{description}
\item[{20 Jul 2014 - Joao Gerd - upgrade to new version of m\_string.f90}] \leavevmode\begin{itemize}
\item {} 
update help banner and documentation

\end{itemize}

\end{description}

\end{itemize}

\end{description}
\begin{quote}\begin{description}
\item[{Use }] \leavevmode
\sphinxcode{m\_stdio}, {\hyperref[\detokenize{m_time:f/time_module}]{\sphinxcrossref{\sphinxcode{time\_module}}}}, {\hyperref[\detokenize{m_string:f/m_string}]{\sphinxcrossref{\sphinxcode{m\_string}}}}

\item[{Call to}] \leavevmode
{\hyperref[\detokenize{inctime:f/_/usage}]{\sphinxcrossref{\sphinxcode{usage()}}}}, {\hyperref[\detokenize{m_time:f/time_module/jul2cal}]{\sphinxcrossref{\sphinxcode{jul2cal()}}}}, {\hyperref[\detokenize{m_time:f/time_module/doy}]{\sphinxcrossref{\sphinxcode{doy()}}}}

\end{description}\end{quote}

\end{fulllineitems}

\paragraph{Subroutines and functions}
\index{usage() (fortran subroutine)|textbf}

\begin{fulllineitems}
\phantomsection\label{\detokenize{inctime:f/_/usage}}\pysiglinewithargsret{\sphinxbfcode{subroutine  }\sphinxbfcode{usage}}{}{}~\begin{quote}\begin{description}
\item[{Use }] \leavevmode
\sphinxcode{m\_stdio}

\item[{Called from}] \leavevmode
{\hyperref[\detokenize{inctime:f/inctime}]{\sphinxcrossref{\sphinxcode{inctime}}}}

\end{description}\end{quote}

\end{fulllineitems}



\subsection{m\_time}
\label{\detokenize{m_time:m-time}}\label{\detokenize{m_time::doc}}
Inctime is the main program.
\paragraph{Module}
\phantomsection\label{\detokenize{m_time:f/time_module}}\index{time\_module (module)|textbf}\paragraph{Description}
\begin{description}
\item[{Description}] \leavevmode
This module contains routines and functions to manipulate time
periods, e.g, functions to calculate total number of hours, days,
months and years between two dates, also contains routines to
convert julian days to gregorian day and vice and versa.

\item[{History}] \leavevmode\begin{itemize}
\item {} 
15 Jun 2005 - J. G. de Mattos - Initial Version

\item {} \begin{description}
\item[{18 Mar 2010 - J. G. de Mattos - Include Time calculation:}] \leavevmode\begin{itemize}
\item {} 
End day of Month {[}eom{]}

\item {} 
Number of hours {[}noh{]}

\item {} 
Number of days {[}nod{]}

\item {} 
Number of months {[}nom{]}

\item {} 
Number of year {[}moy{]}

\end{itemize}

\end{description}

\item {} \begin{description}
\item[{23 Mar 2010 - J. G. de Mattos - Modified the call for Cal2Jul routine}] \leavevmode
was created the interface block for
use of this new Cal2Jul

\end{description}

\item {} 
09 May 2013 - J. G. de Mattos - Removed Bug in Number of Hours

\item {} 
05 Feb 2014 - J. G. de Mattos - Include day of year calculation

\end{itemize}

\end{description}
\paragraph{Quick access}
\begin{quote}\begin{description}
\item[{Variables}] \leavevmode
{\hyperref[\detokenize{m_time:f/time_module/cal2jul}]{\sphinxcrossref{\sphinxcode{cal2jul}}}}

\item[{Routines}] \leavevmode
{\hyperref[\detokenize{m_time:f/time_module/jul2cal}]{\sphinxcrossref{\sphinxcode{jul2cal()}}}}, {\hyperref[\detokenize{m_time:f/time_module/cal2jul_}]{\sphinxcrossref{\sphinxcode{cal2jul\_()}}}}, {\hyperref[\detokenize{m_time:f/time_module/noy}]{\sphinxcrossref{\sphinxcode{noy()}}}}, {\hyperref[\detokenize{m_time:f/time_module/nod}]{\sphinxcrossref{\sphinxcode{nod()}}}}, {\hyperref[\detokenize{m_time:f/time_module/noh}]{\sphinxcrossref{\sphinxcode{noh()}}}}, {\hyperref[\detokenize{m_time:f/time_module/nom}]{\sphinxcrossref{\sphinxcode{nom()}}}}, {\hyperref[\detokenize{m_time:f/time_module/doy}]{\sphinxcrossref{\sphinxcode{doy()}}}}, {\hyperref[\detokenize{m_time:f/time_module/eom}]{\sphinxcrossref{\sphinxcode{eom()}}}}, {\hyperref[\detokenize{m_time:f/time_module/cal2jul__}]{\sphinxcrossref{\sphinxcode{cal2jul\_\_()}}}}

\end{description}\end{quote}
\paragraph{Variables}
\begin{itemize}
\item {} \index{cal2jul (fortran variable in module time\_module)|textbf}

\begin{fulllineitems}
\phantomsection\label{\detokenize{m_time:f/time_module/cal2jul}}\pysigline{\sphinxcode{time\_module/}\sphinxbfcode{cal2jul}}
\end{fulllineitems}

\begin{quote}\begin{description}
\item[{type}] \leavevmode
\item[{attrs}] \leavevmode
public

\end{description}\end{quote}

Convert from gregorian to julian day

\end{itemize}
\paragraph{Subroutines and functions}
\index{eom() (fortran function in module time\_module)|textbf}

\begin{fulllineitems}
\phantomsection\label{\detokenize{m_time:f/time_module/eom}}\pysiglinewithargsret{\sphinxbfcode{function  }\sphinxcode{time\_module/}\sphinxbfcode{eom}}{\emph{year}, \emph{month}}{}~\begin{description}
\item[{Description}] \leavevmode
This function calculate the end day of month.

\item[{History}] \leavevmode\begin{itemize}
\item {} 
18 Mar 2010 - J. G. de Mattos - Initial Version

\end{itemize}

\end{description}
\begin{quote}\begin{description}
\item[{Parameters}] \leavevmode\begin{itemize}
\item {} 
\sphinxstylestrong{year}\sphinxstyleemphasis{ {[}}\sphinxstyleemphasis{integer}\sphinxstyleemphasis{,}\sphinxstyleemphasis{in}\sphinxstyleemphasis{{]}}

\item {} 
\sphinxstylestrong{month}\sphinxstyleemphasis{ {[}}\sphinxstyleemphasis{integer}\sphinxstyleemphasis{,}\sphinxstyleemphasis{in}\sphinxstyleemphasis{{]}}

\end{itemize}

\item[{Return}] \leavevmode
\sphinxstylestrong{day}\sphinxstyleemphasis{ {[}}\sphinxstyleemphasis{integer}\sphinxstyleemphasis{{]}}

\end{description}\end{quote}

\end{fulllineitems}

\index{noh() (fortran function in module time\_module)|textbf}

\begin{fulllineitems}
\phantomsection\label{\detokenize{m_time:f/time_module/noh}}\pysiglinewithargsret{\sphinxbfcode{function  }\sphinxcode{time\_module/}\sphinxbfcode{noh}}{\emph{di}, \emph{df}}{}~\begin{quote}
\begin{description}
\item[{Description}] \leavevmode
This function calculate the total number of hours between two dates.

\item[{History}] \leavevmode\begin{itemize}
\item {} 
18 Mar 2010 - J. G. de Mattos - Initial Version

\end{itemize}

\end{description}
\end{quote}
\begin{quote}\begin{description}
\item[{Parameters}] \leavevmode\begin{itemize}
\item {} 
\sphinxstylestrong{di}\sphinxstyleemphasis{ {[}}\sphinxstyleemphasis{integer}\sphinxstyleemphasis{,}\sphinxstyleemphasis{in}\sphinxstyleemphasis{{]}} :: Starting Date

\item {} 
\sphinxstylestrong{df}\sphinxstyleemphasis{ {[}}\sphinxstyleemphasis{integer}\sphinxstyleemphasis{,}\sphinxstyleemphasis{in}\sphinxstyleemphasis{{]}} :: Ending Date

\end{itemize}

\item[{Return}] \leavevmode
\sphinxstylestrong{nhour}\sphinxstyleemphasis{ {[}}\sphinxstyleemphasis{integer}\sphinxstyleemphasis{{]}}

\end{description}\end{quote}

\end{fulllineitems}

\index{nod() (fortran function in module time\_module)|textbf}

\begin{fulllineitems}
\phantomsection\label{\detokenize{m_time:f/time_module/nod}}\pysiglinewithargsret{\sphinxbfcode{function  }\sphinxcode{time\_module/}\sphinxbfcode{nod}}{\emph{di}, \emph{df}}{}~\begin{quote}
\begin{description}
\item[{Description}] \leavevmode
This function calculate the total number of days between two dates.

\item[{History}] \leavevmode\begin{itemize}
\item {} 
18 Mar 2010 - J. G. de Mattos - Initial Version

\end{itemize}

\end{description}
\end{quote}
\begin{quote}\begin{description}
\item[{Parameters}] \leavevmode\begin{itemize}
\item {} 
\sphinxstylestrong{di}\sphinxstyleemphasis{ {[}}\sphinxstyleemphasis{integer}\sphinxstyleemphasis{,}\sphinxstyleemphasis{in}\sphinxstyleemphasis{{]}} :: Starting Date

\item {} 
\sphinxstylestrong{df}\sphinxstyleemphasis{ {[}}\sphinxstyleemphasis{integer}\sphinxstyleemphasis{,}\sphinxstyleemphasis{in}\sphinxstyleemphasis{{]}} :: Ending Date

\end{itemize}

\item[{Return}] \leavevmode
\sphinxstylestrong{nday}\sphinxstyleemphasis{ {[}}\sphinxstyleemphasis{integer}\sphinxstyleemphasis{{]}}

\end{description}\end{quote}

\end{fulllineitems}

\index{nom() (fortran function in module time\_module)|textbf}

\begin{fulllineitems}
\phantomsection\label{\detokenize{m_time:f/time_module/nom}}\pysiglinewithargsret{\sphinxbfcode{function  }\sphinxcode{time\_module/}\sphinxbfcode{nom}}{\emph{di}, \emph{df}}{}~\begin{quote}
\begin{description}
\item[{Description}] \leavevmode
This function calculate the total number of months between two dates.

\item[{History}] \leavevmode\begin{itemize}
\item {} 
18 Mar 2010 - J. G. de Mattos - Initial Version

\end{itemize}

\end{description}
\end{quote}
\begin{quote}\begin{description}
\item[{Parameters}] \leavevmode\begin{itemize}
\item {} 
\sphinxstylestrong{di}\sphinxstyleemphasis{ {[}}\sphinxstyleemphasis{integer}\sphinxstyleemphasis{,}\sphinxstyleemphasis{in}\sphinxstyleemphasis{{]}} :: Starting Date

\item {} 
\sphinxstylestrong{df}\sphinxstyleemphasis{ {[}}\sphinxstyleemphasis{integer}\sphinxstyleemphasis{,}\sphinxstyleemphasis{in}\sphinxstyleemphasis{{]}} :: Ending Date

\end{itemize}

\item[{Return}] \leavevmode
\sphinxstylestrong{nmonth}\sphinxstyleemphasis{ {[}}\sphinxstyleemphasis{integer}\sphinxstyleemphasis{{]}}

\end{description}\end{quote}

\end{fulllineitems}

\index{noy() (fortran function in module time\_module)|textbf}

\begin{fulllineitems}
\phantomsection\label{\detokenize{m_time:f/time_module/noy}}\pysiglinewithargsret{\sphinxbfcode{function  }\sphinxcode{time\_module/}\sphinxbfcode{noy}}{\emph{di}, \emph{df}}{}~\begin{quote}
\begin{description}
\item[{Description}] \leavevmode
This function calculate the total number of Years between two dates.

\item[{History}] \leavevmode\begin{itemize}
\item {} 
18 Mar 2010 - J. G. de Mattos - Initial Version

\end{itemize}

\end{description}
\end{quote}
\begin{quote}\begin{description}
\item[{Parameters}] \leavevmode\begin{itemize}
\item {} 
\sphinxstylestrong{di}\sphinxstyleemphasis{ {[}}\sphinxstyleemphasis{integer}\sphinxstyleemphasis{,}\sphinxstyleemphasis{in}\sphinxstyleemphasis{{]}} :: Starting Date

\item {} 
\sphinxstylestrong{df}\sphinxstyleemphasis{ {[}}\sphinxstyleemphasis{integer}\sphinxstyleemphasis{,}\sphinxstyleemphasis{in}\sphinxstyleemphasis{{]}} :: Ending Date

\end{itemize}

\item[{Return}] \leavevmode
\sphinxstylestrong{nyear}\sphinxstyleemphasis{ {[}}\sphinxstyleemphasis{integer}\sphinxstyleemphasis{{]}}

\end{description}\end{quote}

\end{fulllineitems}

\index{doy() (fortran function in module time\_module)|textbf}

\begin{fulllineitems}
\phantomsection\label{\detokenize{m_time:f/time_module/doy}}\pysiglinewithargsret{\sphinxbfcode{function  }\sphinxcode{time\_module/}\sphinxbfcode{doy}}{\emph{nymd}, \emph{nhms}}{}~\begin{quote}
\begin{description}
\item[{Description}] \leavevmode
This function calculate the day of the year

\item[{History}] \leavevmode\begin{itemize}
\item {} 
05 Feb 2014 - J. G. de Mattos - Initial Version

\end{itemize}

\end{description}
\end{quote}
\begin{quote}\begin{description}
\item[{Parameters}] \leavevmode\begin{itemize}
\item {} 
\sphinxstylestrong{nymd}\sphinxstyleemphasis{ {[}}\sphinxstyleemphasis{integer}\sphinxstyleemphasis{,}\sphinxstyleemphasis{in}\sphinxstyleemphasis{{]}} :: year month day (yyyymmdd)

\item {} 
\sphinxstylestrong{nhms}\sphinxstyleemphasis{ {[}}\sphinxstyleemphasis{integer}\sphinxstyleemphasis{,}\sphinxstyleemphasis{in}\sphinxstyleemphasis{{]}} :: hour minute second (hhmnsd)

\end{itemize}

\item[{Return}] \leavevmode
\sphinxstylestrong{day}\sphinxstyleemphasis{ {[}}\sphinxstyleemphasis{real}\sphinxstyleemphasis{{]}}

\item[{Called from}] \leavevmode
{\hyperref[\detokenize{inctime:f/inctime}]{\sphinxcrossref{\sphinxcode{inctime}}}}

\item[{Call to}] \leavevmode
{\hyperref[\detokenize{m_time:f/time_module/cal2jul__}]{\sphinxcrossref{\sphinxcode{cal2jul\_\_()}}}}

\end{description}\end{quote}

\end{fulllineitems}

\index{cal2jul\_() (fortran function in module time\_module)|textbf}

\begin{fulllineitems}
\phantomsection\label{\detokenize{m_time:f/time_module/cal2jul_}}\pysiglinewithargsret{\sphinxbfcode{function  }\sphinxcode{time\_module/}\sphinxbfcode{cal2jul\_}}{\emph{caldate}}{}~\begin{quote}
\begin{description}
\item[{Description}] \leavevmode
This function calculate the julian day from gregorian day.

\item[{History}] \leavevmode\begin{itemize}
\item {} 
15 Jun 2005 - J. G. de Mattos - Initial Version

\end{itemize}

\end{description}
\end{quote}
\begin{quote}\begin{description}
\item[{Parameters}] \leavevmode
\sphinxstylestrong{caldate}\sphinxstyleemphasis{ {[}}\sphinxstyleemphasis{integer}\sphinxstyleemphasis{,}\sphinxstyleemphasis{in}\sphinxstyleemphasis{{]}}

\item[{Return}] \leavevmode
\sphinxstylestrong{julian}\sphinxstyleemphasis{ {[}}\sphinxstyleemphasis{real}\sphinxstyleemphasis{{]}}

\item[{Call to}] \leavevmode
{\hyperref[\detokenize{m_time:f/time_module/cal2jul__}]{\sphinxcrossref{\sphinxcode{cal2jul\_\_()}}}}

\end{description}\end{quote}

\end{fulllineitems}

\index{cal2jul\_\_() (fortran function in module time\_module)|textbf}

\begin{fulllineitems}
\phantomsection\label{\detokenize{m_time:f/time_module/cal2jul__}}\pysiglinewithargsret{\sphinxbfcode{function  }\sphinxcode{time\_module/}\sphinxbfcode{cal2jul\_\_}}{\emph{ymd}, \emph{hms}}{}~\begin{quote}
\begin{description}
\item[{Description}] \leavevmode
This function calculate the julian day from gregorian day

\item[{History}] \leavevmode\begin{itemize}
\item {} 
15 Jun 2005 - J. G. de Mattos - Initial Version

\end{itemize}

\end{description}
\end{quote}
\begin{quote}\begin{description}
\item[{Parameters}] \leavevmode\begin{itemize}
\item {} 
\sphinxstylestrong{ymd}\sphinxstyleemphasis{ {[}}\sphinxstyleemphasis{integer}\sphinxstyleemphasis{,}\sphinxstyleemphasis{in}\sphinxstyleemphasis{{]}}

\item {} 
\sphinxstylestrong{hms}\sphinxstyleemphasis{ {[}}\sphinxstyleemphasis{integer}\sphinxstyleemphasis{,}\sphinxstyleemphasis{in}\sphinxstyleemphasis{{]}}

\end{itemize}

\item[{Return}] \leavevmode
\sphinxstylestrong{julian}\sphinxstyleemphasis{ {[}}\sphinxstyleemphasis{real}\sphinxstyleemphasis{{]}}

\item[{Called from}] \leavevmode
{\hyperref[\detokenize{m_time:f/time_module/doy}]{\sphinxcrossref{\sphinxcode{doy()}}}}, {\hyperref[\detokenize{m_time:f/time_module/cal2jul_}]{\sphinxcrossref{\sphinxcode{cal2jul\_()}}}}

\end{description}\end{quote}

\end{fulllineitems}

\index{jul2cal() (fortran subroutine in module time\_module)|textbf}

\begin{fulllineitems}
\phantomsection\label{\detokenize{m_time:f/time_module/jul2cal}}\pysiglinewithargsret{\sphinxbfcode{subroutine  }\sphinxcode{time\_module/}\sphinxbfcode{jul2cal}}{\emph{jd}, \emph{ymd}, \emph{hms}}{}~\begin{quote}
\begin{description}
\item[{Description}] \leavevmode
This function calculate the gregorian date from julian day.

\item[{History}] \leavevmode\begin{itemize}
\item {} 
15 Jun 2005 - J. G. de Mattos - Initial Version

\item {} \begin{description}
\item[{23 Mar 2011 - J. G. de Mattos - Modified Interface}] \leavevmode
to a subroutine call

\end{description}

\end{itemize}

\item[{Remarks}] \leavevmode
This algorithm was adopted from Press et al.

\end{description}
\end{quote}
\begin{quote}\begin{description}
\item[{Parameters}] \leavevmode\begin{itemize}
\item {} 
\sphinxstylestrong{jd}\sphinxstyleemphasis{ {[}}\sphinxstyleemphasis{real}\sphinxstyleemphasis{,}\sphinxstyleemphasis{in}\sphinxstyleemphasis{{]}}

\item {} 
\sphinxstylestrong{ymd}\sphinxstyleemphasis{ {[}}\sphinxstyleemphasis{integer}\sphinxstyleemphasis{{]}} :: year month day (yyyymmdd)

\item {} 
\sphinxstylestrong{hms}\sphinxstyleemphasis{ {[}}\sphinxstyleemphasis{integer}\sphinxstyleemphasis{{]}} :: hour minute second (hhmnsd)

\end{itemize}

\item[{Called from}] \leavevmode
{\hyperref[\detokenize{inctime:f/inctime}]{\sphinxcrossref{\sphinxcode{inctime}}}}

\end{description}\end{quote}

\end{fulllineitems}



\subsection{m\_string}
\label{\detokenize{m_string::doc}}\label{\detokenize{m_string:m-string}}
Inctime is the main program.
\paragraph{Module}
\phantomsection\label{\detokenize{m_string:f/m_string}}\index{m\_string (module)|textbf}\paragraph{Description}

A module to process strings.
\begin{description}
\item[{Description}] \leavevmode
Make some operations in strings

\item[{History}] \leavevmode\begin{itemize}
\item {} 
15 Dec 2010 - J. G. de Mattos - Initial Version

\item {} 
02 Mar 2011 - J. G. de Mattos - Initial code to strTemplate

\item {} 
30 Nov 2012 - J. G. de Mattos - All input parameters optionals in strTemplate

\item {} 
05 Fev 2014 - J. G. de Mattos - Adding julian day mask

\item {} 
20 jun 2014 - J. G. de Mattos - Adding GetTokens feature

\end{itemize}

\end{description}
\paragraph{Quick access}
\begin{quote}\begin{description}
\item[{Variables}] \leavevmode
{\hyperref[\detokenize{m_string:f/m_string/gettokens}]{\sphinxcrossref{\sphinxcode{gettokens}}}}, {\hyperref[\detokenize{m_string:f/m_string/replace}]{\sphinxcrossref{\sphinxcode{replace}}}}, {\hyperref[\detokenize{m_string:f/m_string/str_template}]{\sphinxcrossref{\sphinxcode{str\_template}}}}, {\hyperref[\detokenize{m_string:f/m_string/num2str}]{\sphinxcrossref{\sphinxcode{num2str}}}}, {\hyperref[\detokenize{m_string:f/m_string/mon_lc}]{\sphinxcrossref{\sphinxcode{mon\_lc}}}}, {\hyperref[\detokenize{m_string:f/m_string/mon_uc}]{\sphinxcrossref{\sphinxcode{mon\_uc}}}}, {\hyperref[\detokenize{m_string:f/m_string/mon_wd}]{\sphinxcrossref{\sphinxcode{mon\_wd}}}}

\item[{Routines}] \leavevmode
{\hyperref[\detokenize{m_string:f/m_string/gettokens_}]{\sphinxcrossref{\sphinxcode{gettokens\_()}}}}, {\hyperref[\detokenize{m_string:f/m_string/float2str}]{\sphinxcrossref{\sphinxcode{float2str()}}}}, {\hyperref[\detokenize{m_string:f/m_string/int2str}]{\sphinxcrossref{\sphinxcode{int2str()}}}}, {\hyperref[\detokenize{m_string:f/m_string/replace_}]{\sphinxcrossref{\sphinxcode{replace\_()}}}}, {\hyperref[\detokenize{m_string:f/m_string/str_template_}]{\sphinxcrossref{\sphinxcode{str\_template\_()}}}}

\end{description}\end{quote}
\paragraph{Variables}
\begin{itemize}
\item {} \index{mon\_wd (fortran variable in module m\_string)|textbf}

\begin{fulllineitems}
\phantomsection\label{\detokenize{m_string:f/m_string/mon_wd}}\pysigline{\sphinxcode{m\_string/}\sphinxbfcode{mon\_wd}}
\end{fulllineitems}

\begin{quote}\begin{description}
\item[{shape}] \leavevmode\begin{enumerate}
\setcounter{enumi}{11}
\item {} 
\end{enumerate}

\item[{type}] \leavevmode
character

\item[{attrs}] \leavevmode
private/parameter=(/’jan’,’feb’,’mar’,’apr’,’may’,’jun’,’jul’,’aug’,’sep’,’oct’,’nov’,’dec’/)

\end{description}\end{quote}

\item {} \index{gettokens (fortran variable in module m\_string)|textbf}

\begin{fulllineitems}
\phantomsection\label{\detokenize{m_string:f/m_string/gettokens}}\pysigline{\sphinxcode{m\_string/}\sphinxbfcode{gettokens}}
\end{fulllineitems}

\begin{quote}\begin{description}
\item[{type}] \leavevmode
\item[{attrs}] \leavevmode
public

\end{description}\end{quote}

Get tokens by line

\item {} \index{replace (fortran variable in module m\_string)|textbf}

\begin{fulllineitems}
\phantomsection\label{\detokenize{m_string:f/m_string/replace}}\pysigline{\sphinxcode{m\_string/}\sphinxbfcode{replace}}
\end{fulllineitems}

\begin{quote}\begin{description}
\item[{type}] \leavevmode
\item[{attrs}] \leavevmode
public

\end{description}\end{quote}

Replace a string by another

\item {} \index{str\_template (fortran variable in module m\_string)|textbf}

\begin{fulllineitems}
\phantomsection\label{\detokenize{m_string:f/m_string/str_template}}\pysigline{\sphinxcode{m\_string/}\sphinxbfcode{str\_template}}
\end{fulllineitems}

\begin{quote}\begin{description}
\item[{type}] \leavevmode
\item[{attrs}] \leavevmode
public

\end{description}\end{quote}

Replace variables in a template

\item {} \index{mon\_uc (fortran variable in module m\_string)|textbf}

\begin{fulllineitems}
\phantomsection\label{\detokenize{m_string:f/m_string/mon_uc}}\pysigline{\sphinxcode{m\_string/}\sphinxbfcode{mon\_uc}}
\end{fulllineitems}

\begin{quote}\begin{description}
\item[{shape}] \leavevmode\begin{enumerate}
\setcounter{enumi}{11}
\item {} 
\end{enumerate}

\item[{type}] \leavevmode
character

\item[{attrs}] \leavevmode
private/parameter=(/’jan’,’feb’,’mar’,’apr’,’may’,’jun’,’jul’,’aug’,’sep’,’oct’,’nov’,’dec’/)

\end{description}\end{quote}

\item {} \index{mon\_lc (fortran variable in module m\_string)|textbf}

\begin{fulllineitems}
\phantomsection\label{\detokenize{m_string:f/m_string/mon_lc}}\pysigline{\sphinxcode{m\_string/}\sphinxbfcode{mon\_lc}}
\end{fulllineitems}

\begin{quote}\begin{description}
\item[{shape}] \leavevmode\begin{enumerate}
\setcounter{enumi}{11}
\item {} 
\end{enumerate}

\item[{type}] \leavevmode
character

\item[{attrs}] \leavevmode
private/parameter=(/’jan’,’feb’,’mar’,’apr’,’may’,’jun’,’jul’,’aug’,’sep’,’oct’,’nov’,’dec’/)

\end{description}\end{quote}

\item {} \index{num2str (fortran variable in module m\_string)|textbf}

\begin{fulllineitems}
\phantomsection\label{\detokenize{m_string:f/m_string/num2str}}\pysigline{\sphinxcode{m\_string/}\sphinxbfcode{num2str}}
\end{fulllineitems}

\begin{quote}\begin{description}
\item[{type}] \leavevmode
\item[{attrs}] \leavevmode
public

\end{description}\end{quote}

convert a number to string

\end{itemize}
\paragraph{Subroutines and functions}
\index{str\_template\_() (fortran subroutine in module m\_string)|textbf}

\begin{fulllineitems}
\phantomsection\label{\detokenize{m_string:f/m_string/str_template_}}\pysiglinewithargsret{\sphinxbfcode{subroutine  }\sphinxcode{m\_string/}\sphinxbfcode{str\_template\_}}{\emph{strg}\sphinxoptional{, \emph{nymd}\sphinxoptional{, \emph{nhms}\sphinxoptional{, \emph{fymd}\sphinxoptional{, \emph{fhms}\sphinxoptional{, \emph{jd}\sphinxoptional{, \emph{doy}\sphinxoptional{, \emph{label}}}}}}}}}{}~\begin{quote}

A template formatting a string with variables.
\begin{description}
\item[{Description}] \leavevmode
A template resolver formatting a string with a string variable
and time variables.  The format descriptors are similar to those
used in the GrADS.

\item[{Variables}] \leavevmode\begin{itemize}
\item {} 
\%y4    substitute with a 4 digit year

\item {} 
\%y2    a 2 digit year

\item {} 
\%m1    a 1 or 2 digit month

\item {} 
\%m2    a 2 digit month

\item {} 
\%mc    a 3 letter month in lower cases

\item {} 
\%Mc    a 3 letter month with a leading letter in upper case

\item {} 
\%MC    a 3 letter month in upper cases

\item {} 
\%d1    a 1 or 2 digit day

\item {} 
\%d2    a 2 digit day

\item {} 
\%h1    a 1 or 2 digit hour

\item {} 
\%h2    a 2 digit hour

\item {} 
\%h3    a 3 digit hour (?)

\item {} 
\%n2    a 2 digit minute

\item {} 
\%e     a string ensemble identify

\item {} 
\%jd    a julian day without hours decimals

\item {} 
\%jdh   a julian day with hour decimals

\item {} 
\%jy    a day of current year without hours decimals

\item {} 
\%jyh   a day of current year with hours decimals

\item {} 
\%ix1   initial 1 digit decade

\item {} 
\%ix3   initial 3 digit decade

\item {} 
\%iy2   initial 2 digit year

\item {} 
\%iy4   initial 4 digit year

\item {} 
\%im1   initial 1 or 2 digit month

\item {} 
\%im2   initial 2 digit month (leading zero if needed)

\item {} 
\%imc   initial 3 character month abbreviation

\item {} 
\%id1   initial 1 or 2 digit day (leading zero if needed)

\item {} 
\%id2   initial 2 digit day

\item {} 
\%ih1   initial 1 or 2 digit hour

\item {} 
\%ih2   initial 2 digit hour

\item {} 
\%ih3   initial 3 digit hour

\item {} 
\%in2   initial 2 digit minute (leading zero if needed)

\item {} 
\%fx1   forecast 1 digit decade

\item {} 
\%fx3   forecast 3 digit decade

\item {} 
\%fy2   forecast 2 digit year

\item {} 
\%fy4   forecast 4 digit year

\item {} 
\%fm1   forecast 1 or 2 digit month

\item {} 
\%fm2   forecast 2 digit month (leading zero if needed)

\item {} 
\%fmc   forecast 3 character month abbreviation

\item {} 
\%fd1   forecast 1 or 2 digit day (leading zero if needed)

\item {} 
\%fd2   forecast 2 digit day

\item {} 
\%fh1   forecast 1 or 2 digit hour

\item {} 
\%fh2   forecast 2 digit hour

\item {} 
\%fh3   forecast 3 digit hour

\item {} 
\%fn2   forecast 2 digit minute (leading zero if needed)

\end{itemize}

\item[{History}] \leavevmode\begin{itemize}
\item {} 
Joao Gerd - 02Mar2011 - Codigo Inicial

\item {} 
Joao Gerd - 30Nov2012 - All input parameters optionals in strTemplate

\item {} 
Joao Gerd - 05Fev2014 - Adding julian day mask

\end{itemize}

\end{description}
\end{quote}
\begin{quote}\begin{description}
\item[{Parameters}] \leavevmode
\sphinxstylestrong{strg}\sphinxstyleemphasis{ {[}}\sphinxstyleemphasis{character}\sphinxstyleemphasis{,}\sphinxstyleemphasis{inout}\sphinxstyleemphasis{{]}}

\item[{Options}] \leavevmode\begin{itemize}
\item {} 
\sphinxstylestrong{nymd}\sphinxstyleemphasis{ {[}}\sphinxstyleemphasis{integer}\sphinxstyleemphasis{,}\sphinxstyleemphasis{in,optional}\sphinxstyleemphasis{{]}}

\item {} 
\sphinxstylestrong{nhms}\sphinxstyleemphasis{ {[}}\sphinxstyleemphasis{integer}\sphinxstyleemphasis{,}\sphinxstyleemphasis{in,optional}\sphinxstyleemphasis{{]}}

\item {} 
\sphinxstylestrong{fymd}\sphinxstyleemphasis{ {[}}\sphinxstyleemphasis{integer}\sphinxstyleemphasis{,}\sphinxstyleemphasis{in,optional}\sphinxstyleemphasis{{]}}

\item {} 
\sphinxstylestrong{fhms}\sphinxstyleemphasis{ {[}}\sphinxstyleemphasis{integer}\sphinxstyleemphasis{,}\sphinxstyleemphasis{in,optional}\sphinxstyleemphasis{{]}}

\item {} 
\sphinxstylestrong{jd}\sphinxstyleemphasis{ {[}}\sphinxstyleemphasis{real}\sphinxstyleemphasis{,}\sphinxstyleemphasis{in,optional}\sphinxstyleemphasis{{]}}

\item {} 
\sphinxstylestrong{doy}\sphinxstyleemphasis{ {[}}\sphinxstyleemphasis{real}\sphinxstyleemphasis{,}\sphinxstyleemphasis{in,optional}\sphinxstyleemphasis{{]}}

\item {} 
\sphinxstylestrong{label}\sphinxstyleemphasis{ {[}}\sphinxstyleemphasis{character}\sphinxstyleemphasis{,}\sphinxstyleemphasis{in,optional}\sphinxstyleemphasis{{]}}

\end{itemize}

\item[{Call to}] \leavevmode
{\hyperref[\detokenize{m_string:f/m_string/replace_}]{\sphinxcrossref{\sphinxcode{replace\_()}}}}, {\hyperref[\detokenize{m_string:f/m_string/int2str}]{\sphinxcrossref{\sphinxcode{int2str()}}}}, {\hyperref[\detokenize{m_string:f/m_string/float2str}]{\sphinxcrossref{\sphinxcode{float2str()}}}}

\end{description}\end{quote}

\end{fulllineitems}

\index{replace\_() (fortran subroutine in module m\_string)|textbf}

\begin{fulllineitems}
\phantomsection\label{\detokenize{m_string:f/m_string/replace_}}\pysiglinewithargsret{\sphinxbfcode{subroutine  }\sphinxcode{m\_string/}\sphinxbfcode{replace\_}}{\emph{strg}, \emph{mask}, \emph{repl}}{}~\begin{quote}
\begin{description}
\item[{Description}] \leavevmode
Rotina para substituir a mask pela repl na strg.

\item[{History}] \leavevmode\begin{itemize}
\item {} 
Joao Gerd - 20Feb2011 - Codigo Inicial.

\end{itemize}

\end{description}
\end{quote}
\begin{quote}\begin{description}
\item[{Parameters}] \leavevmode\begin{itemize}
\item {} 
\sphinxstylestrong{strg}\sphinxstyleemphasis{ {[}}\sphinxstyleemphasis{character}\sphinxstyleemphasis{,}\sphinxstyleemphasis{inout}\sphinxstyleemphasis{{]}} :: String

\item {} 
\sphinxstylestrong{mask}\sphinxstyleemphasis{ {[}}\sphinxstyleemphasis{character}\sphinxstyleemphasis{,}\sphinxstyleemphasis{in}\sphinxstyleemphasis{{]}} :: maskout

\item {} 
\sphinxstylestrong{repl}\sphinxstyleemphasis{ {[}}\sphinxstyleemphasis{character}\sphinxstyleemphasis{,}\sphinxstyleemphasis{in}\sphinxstyleemphasis{{]}} :: replacing string

\end{itemize}

\item[{Called from}] \leavevmode
{\hyperref[\detokenize{m_string:f/m_string/str_template_}]{\sphinxcrossref{\sphinxcode{str\_template\_()}}}}

\end{description}\end{quote}

\end{fulllineitems}

\index{gettokens\_() (fortran subroutine in module m\_string)|textbf}

\begin{fulllineitems}
\phantomsection\label{\detokenize{m_string:f/m_string/gettokens_}}\pysiglinewithargsret{\sphinxbfcode{subroutine  }\sphinxcode{m\_string/}\sphinxbfcode{gettokens\_}}{\emph{line}, \emph{tokens}, \emph{ntokens}\sphinxoptional{, \emph{del}}}{}~\begin{quote}\begin{description}
\item[{Parameters}] \leavevmode\begin{itemize}
\item {} 
\sphinxstylestrong{line}\sphinxstyleemphasis{ {[}}\sphinxstyleemphasis{character}\sphinxstyleemphasis{{]}}

\item {} 
\sphinxstylestrong{tokens} (*)\sphinxstyleemphasis{ {[}}\sphinxstyleemphasis{character}\sphinxstyleemphasis{{]}}

\item {} 
\sphinxstylestrong{ntokens}\sphinxstyleemphasis{ {[}}\sphinxstyleemphasis{integer}\sphinxstyleemphasis{{]}}

\end{itemize}

\item[{Options}] \leavevmode
\sphinxstylestrong{del}\sphinxstyleemphasis{ {[}}\sphinxstyleemphasis{character}\sphinxstyleemphasis{,}\sphinxstyleemphasis{optional}\sphinxstyleemphasis{{]}}

\end{description}\end{quote}

\end{fulllineitems}

\index{int2str() (fortran function in module m\_string)|textbf}

\begin{fulllineitems}
\phantomsection\label{\detokenize{m_string:f/m_string/int2str}}\pysiglinewithargsret{\sphinxbfcode{function  }\sphinxcode{m\_string/}\sphinxbfcode{int2str}}{\emph{num}, \emph{format}}{}~\begin{quote}\begin{description}
\item[{Parameters}] \leavevmode\begin{itemize}
\item {} 
\sphinxstylestrong{num}\sphinxstyleemphasis{ {[}}\sphinxstyleemphasis{integer}\sphinxstyleemphasis{,}\sphinxstyleemphasis{in}\sphinxstyleemphasis{{]}}

\item {} 
\sphinxstylestrong{format}\sphinxstyleemphasis{ {[}}\sphinxstyleemphasis{character}\sphinxstyleemphasis{,}\sphinxstyleemphasis{in}\sphinxstyleemphasis{{]}}

\end{itemize}

\item[{Return}] \leavevmode
\sphinxstylestrong{int2str}\sphinxstyleemphasis{ {[}}\sphinxstyleemphasis{character}\sphinxstyleemphasis{{]}}

\item[{Called from}] \leavevmode
{\hyperref[\detokenize{m_string:f/m_string/str_template_}]{\sphinxcrossref{\sphinxcode{str\_template\_()}}}}

\end{description}\end{quote}

\end{fulllineitems}

\index{float2str() (fortran function in module m\_string)|textbf}

\begin{fulllineitems}
\phantomsection\label{\detokenize{m_string:f/m_string/float2str}}\pysiglinewithargsret{\sphinxbfcode{function  }\sphinxcode{m\_string/}\sphinxbfcode{float2str}}{\emph{num}, \emph{format}}{}~\begin{quote}\begin{description}
\item[{Parameters}] \leavevmode\begin{itemize}
\item {} 
\sphinxstylestrong{num}\sphinxstyleemphasis{ {[}}\sphinxstyleemphasis{real}\sphinxstyleemphasis{,}\sphinxstyleemphasis{in}\sphinxstyleemphasis{{]}}

\item {} 
\sphinxstylestrong{format}\sphinxstyleemphasis{ {[}}\sphinxstyleemphasis{character}\sphinxstyleemphasis{,}\sphinxstyleemphasis{in}\sphinxstyleemphasis{{]}}

\end{itemize}

\item[{Return}] \leavevmode
\sphinxstylestrong{float2str}\sphinxstyleemphasis{ {[}}\sphinxstyleemphasis{character}\sphinxstyleemphasis{{]}}

\item[{Called from}] \leavevmode
{\hyperref[\detokenize{m_string:f/m_string/str_template_}]{\sphinxcrossref{\sphinxcode{str\_template\_()}}}}

\end{description}\end{quote}

\end{fulllineitems}



\subsection{m\_stdio}
\label{\detokenize{m_stdio:m-stdio}}\label{\detokenize{m_stdio::doc}}
A F90 module defines std. I/O parameters.


\chapter{Usage}
\label{\detokenize{index:usage}}

\section{Usage}
\label{\detokenize{usage:usage}}\label{\detokenize{usage::doc}}
In this page is given an overview on how to use inctime.


\subsection{Download}
\label{\detokenize{usage:download}}
The code is hosted in the Redmine portal at CPTEC. To checkout the code, use the following command:

\fvset{hllines={, ,}}%
\begin{sphinxVerbatim}[commandchars=\\\{\}]
\PYGZdl{} svn checkout https://svn.cptec.inpe.br/gdad/jgerd/tags/inctime
\end{sphinxVerbatim}


\subsection{Compile}
\label{\detokenize{usage:compile}}
All is needed is a fortran compiler. To compile the code, enter into the \sphinxcode{src} directory and type \sphinxcode{make}. The \sphinxcode{Makefile} will also compile the associated modules.

Once the compilation is done, an executable called \sphinxcode{inctime} is created.


\subsection{Use}
\label{\detokenize{usage:use}}
The way to use this program is through the command line as follows:

\fvset{hllines={, ,}}%
\begin{sphinxVerbatim}[commandchars=\\\{\}]
\PYGZdl{} ./inctime \PYG{o}{[}yyyymmddhh, yyyymmdd\PYG{o}{]} \PYG{o}{[}\PYGZlt{}+,\PYGZhy{}\PYGZgt{}nynmndnhnnns\PYG{o}{]} \PYG{o}{[}Form. output\PYG{o}{]}
\end{sphinxVerbatim}

The inctime parameters are as follows:
\begin{itemize}
\item {} 
{[}yyyymmddhh, yyyymmdd{]}
\begin{itemize}
\item {} 
Initial Time

\end{itemize}

\item {} 
{[}\textless{}+,-\textgreater{}nynmndnhnnns{]}
\begin{itemize}
\item {} 
( -) calculate the passed date

\item {} 
( +) calculate the future date (default)

\item {} 
(ny) Number of year (default is 0)

\item {} 
(nm) Number of months (default is 0)

\item {} 
(nd) Number of days (default is 0)

\item {} 
(nh) Number of hours (default is 0)

\item {} 
(nn) Number of minutes (default is 0)

\item {} 
(ns) Number of seconds (default is 0)

\end{itemize}

\item {} 
{[} Form. Output {]}
\begin{itemize}
\item {} 
Format to output date. is a template Format
The format descriptors are similar to those
used in the GrADS:
\begin{itemize}
\item {} 
“\%y4”  substitute with a 4 digit year

\item {} 
“\%y2”  a 2 digit year

\item {} 
“\%m1”  a 1 or 2 digit month

\item {} 
“\%m2”  a 2 digit month

\item {} 
“\%mc”  a 3 letter month in lower cases

\item {} 
“\%Mc”  a 3 letter month with a leading letter in upper case

\item {} 
“\%MC”  a 3 letter month in upper cases

\item {} 
“\%d1”  a 1 or 2 digit day

\item {} 
“\%d2”  a 2 digit day

\item {} 
“\%h1”  a 1 or 2 digit hour

\item {} 
“\%h2”  a 2 digit hour

\item {} 
“\%h3”  a 3 digit hour (?)

\item {} 
“\%n2”  a 2 digit minute

\item {} 
“\%e”   a string ensemble identify

\item {} 
“\%jd”  a julian day without hours decimals

\item {} 
“\%jdh” a julian day with hour decimals

\item {} 
“\%jy”  a day of current year without hours decimals

\item {} 
“\%jyh” a day of current year with hours decimals

\end{itemize}

\end{itemize}

\end{itemize}


\sphinxstrong{See also:}


It is possible to use words to compose the output format.



More examples:

\fvset{hllines={, ,}}%
\begin{sphinxVerbatim}[commandchars=\\\{\}]
\PYGZdl{} ./inctime \PYG{l+m}{2001091000} +1d \PYGZpc{}d2/\PYGZpc{}m2/\PYGZpc{}y4
\PYGZdl{} ./inctime \PYG{l+m}{2001091000} +48h30n \PYGZpc{}h2Z\PYGZpc{}d2\PYGZpc{}MC\PYGZpc{}y4
\PYGZdl{} ./inctime \PYG{l+m}{2001091000} \PYGZhy{}1h30n 3B42RT.\PYGZpc{}y4\PYGZpc{}m2\PYGZpc{}d2\PYGZpc{}h2.bin
\PYGZdl{} ./inctime \PYG{l+m}{2001091000} \PYGZhy{}2h45n 3B42RT.\PYGZpc{}y4\PYGZpc{}m2\PYGZpc{}d2\PYGZpc{}h2.bin
\PYGZdl{} ./inctime \PYG{l+m}{2001091000} \PYGZhy{}1y3m2d1h45n ANYTHING.\PYGZpc{}y4\PYGZpc{}m2\PYGZpc{}d2\PYGZpc{}h2.ANYTHING
\end{sphinxVerbatim}


\section{Indices and tables}
\label{\detokenize{index:indices-and-tables}}\begin{itemize}
\item {} 
\DUrole{xref,std,std-ref}{genindex}

\item {} 
\DUrole{xref,std,std-ref}{modindex}

\item {} 
\DUrole{xref,std,std-ref}{search}

\end{itemize}


\renewcommand{\indexname}{Fortran Module Index}
\begin{sphinxtheindex}
\def\bigletter#1{{\Large\sffamily#1}\nopagebreak\vspace{1mm}}
\bigletter{m}
\item {\sphinxstyleindexentry{m\_string}}\sphinxstyleindexpageref{m_string:\detokenize{f/m_string}}
\indexspace
\bigletter{t}
\item {\sphinxstyleindexentry{time\_module}}\sphinxstyleindexpageref{m_time:\detokenize{f/time_module}}
\end{sphinxtheindex}

\renewcommand{\indexname}{Index}
\printindex
\end{document}